% !TEX TS-program = xelatex
% !TEX encoding = UTF-8 Unicode

\documentclass[10pt, a4paper]{article}
\usepackage{fontspec}

% Set up fonts
\usepackage[usenames,dvipsnames]{xcolor}
\usepackage{xunicode}
\usepackage{xltxtra}
\defaultfontfeatures{Mapping=tex-text}
\setromanfont{Linux Libertine O}
\setmonofont[Scale=0.8]{Monaco}

% Layout
\usepackage[xetex]{geometry}
\geometry{a4paper, textwidth=5.5in, textheight=8.5in, marginparsep=7pt, marginparwidth=.6in}
\setlength\parindent{0in}

% Links
\chardef\&="E050
\newcommand{\html}[1]{\href{#1}{\scriptsize\textsc{[html]}}}
\newcommand{\pdf}[1]{\href{#1}{\scriptsize\textsc{[pdf]}}}
\newcommand{\doi}[1]{\href{#1}{\scriptsize\textsc{[doi]}}}

% Margin
\usepackage{marginnote}
\newcommand{\amper{}}{\chardef\amper="E0BD }
\newcommand{\years}[1]{\marginnote{\scriptsize #1}}
\renewcommand*{\raggedleftmarginnote}{}
\setlength{\marginparsep}{7pt}
\reversemarginpar

% Headings
\usepackage{sectsty}
\usepackage[normalem]{ulem}
\sectionfont{\mdseries\upshape\Large}
\subsectionfont{\mdseries\scshape\normalsize}
\subsubsectionfont{\mdseries\upshape\large}
\renewcommand*\thesubsection{\arabic{subsection}.}

% PDF information
\usepackage[
bookmarks, colorlinks, breaklinks, unicode,
pdftitle={Jeremiah Via - vita},
pdfauthor={Jeremiah Via},
pdfproducer={http://jeremiahvia.com}
]{hyperref}
\hypersetup{linkcolor=blue,citecolor=blue,filecolor=black,urlcolor=MidnightBlue}


%% ------------------------------------------------------------------%%
%% Document                                                          %%
%% ------------------------------------------------------------------%%
\begin{document}

%% ------------------------------------------------------------------%%
%% Heading                                                           %%
%% ------------------------------------------------------------------%%
{\LARGE Jeremiah M. Via}\\[1cm]
\begin{minipage}[t]{0.55\textwidth}
  \texttt{571} Bristol Road\\
  Birmingham, \texttt{B29 6AF}\\
  United Kingdom
\end{minipage}
\begin{minipage}[t]{0.4\textwidth}
  Telephone: \texttt{+44 795-607-7116}\\
  Email: \href{mailto:jeremiah.via@gmail.com}{jeremiah.via@gmail.com}\\
  \textsc{url}: \href{http://jeremiahvia.com}{http://jeremiahvia.com}
\end{minipage}


%% ------------------------------------------------------------------%%
%% Area of Specialization & Competence                               %%
%% ------------------------------------------------------------------%%
%% Areas of specialization
%% Areas of competence

%% ------------------------------------------------------------------%%
%% Education                                                         %%
%% ------------------------------------------------------------------%%
\section*{Education}

\years{2012}
\textbf{BSc in Artificial Intelligence \& Computer Science} (First Class Honours)\\
\textsl{University of Birmingham}, United Kingdom\\
\small{\textsc{Advisors:} Nick Hawes; Jeremy Wyatt}\\[.2cm]
Key topics:
\begin{itemize}
\item \textbf{machine learning}: q-learning, probabilistic latent
  semantic analysis, independent component analysis, decision tree
  learning, k-nearest neighbors, case-based reasoning, support vector
  machines
\item \textbf{natural computation}: game theory, cellular automta, ant
  colony optimization, random walks, evolutionary algorithms,
  market-based control, self-orgsnizing maps, particle swarm
  optimization
\item \textbf{natural language processing}: morphological analysis,
  pos tagging, DCG parsing, active chart parsing, recursive transition
  networks, quasi-logical forms
\item \textbf{computer vision}: edge detection, noise filtering, hough
  transform, eigenfaces, object recognition, feature detection
\item \textbf{robotics}: control theory, markov decision processes,
  behavior-based control, probabilistic road maps, particle filtering
\item \textbf{intelligent data analysis}: principle component
  analysis, self-organizing maps, model-based data clustering, latent
  semantic indexing, PageRank
\item \textbf{neural computation}: hebbian learning, gradient descent
  learning, back-propagation, conjugate gradient learning, recurrent
  neural networks, radial basis function networks, self organizing
  maps, learning vector quantization, committee machines, mixture
  models
\end{itemize}
\vspace{.25cm}

\years{2009}
\textbf{AS in Computer Programming} (3.83 GPA)\\
\textit{Grossmont College}, USA\\[.2cm]
Key topics:
\begin{itemize}
\item programming langages: java, c++, x86 assembly
\item data structures: binary trees, heap-tree, graphs, sets,
  b-trees, tries
\item unix
\item software engineering
\end{itemize}

%% ------------------------------------------------------------------%%
%% Appointments                                                      %%
%% ------------------------------------------------------------------%%
\section*{Appointments Held}
\subsection{Research}
\years{2011}
\textbf{Universität Bielefeld}, \textsl{Research Institute for Cognition and Robotics}\\
Summer Research Intern\\[.2cm]
Worked with a PhD student to incoporate a data-driven fault-detection algorithm into the CoSy Architecture Schema Toolkit (CAST).\\[.2cm]
\textbf{University of Birmingham}, \textsl{Intelligent Robotics Lab}\\
Summer Research Intern\\[.2cm]
Performed experiments on robots running CAST in order to determine the efficacy of a data-driven fault-detection algorithm on event-based systems. Results were then used to improve the algorithm.\\[.25cm]

\subsection{Teaching}
\years{2012}
\textbf{University of Birmingham}, \textsl{Robot Programming}\\
Teaching Assistant


%% ------------------------------------------------------------------%%
%% Grants and Awards                                                 %%
%% ------------------------------------------------------------------%%
\section*{Grants \& Awards}
\years{2011}
Ede \& Ravenscroft Travel Bursary\\
Student Development Scholarship\\
Nuffield Foundation Science Bursary\\
School of Computer Science Excellency Scholarship\\
\years{2010}
British Computing Society Tammal Hussein Memorial Prize\\
School of Computer Science Excellency Scholarship\\
\years{2009}
Best First Year Computer Science Student\\
School of Computer Science Excellency Scholarship\\
President's List (4.0 GPA)\\
\years{2008}
Vice-President's List (3.5+ GPA)\\
Vice-President's List (3.5+ GPA)\\
\years{2007}
Vice-President's List (3.5+ GPA)

%% ------------------------------------------------------------------%%
%% Activities                                                        %%
%% ------------------------------------------------------------------%%
\section*{Activities}
\years{2011--2012}\textbf{Vice-Chancellor Seminar Series}\\[.2cm]
Took part in a series of seminar discussions with members from the
other schools of the university. The three top students from each of
the five schools were selected to participate and each school hosted
one discussion topic. Host schools and topics:\\[.1cm]
\small{
  \begin{tabular}{l p{8cm}}
    \hspace{.5cm}Arts \& Law & Is high culture necessarily elitist?\\
    \hspace{.5cm}Social Sciences & Can we still afford the welfare state?\\
    \hspace{.5cm}Engineering and Physical Sciences & Should science spend less time on discovery and more on applying known science for economic benefit?\\
    \hspace{.5cm}Life and Environmental Sciences & Is the opposition to GMOs based on simple scientific ignorance?\\
    \hspace{.5cm}Medical and Dental Sciences & Should life be extended at all costs?
  \end{tabular}
}
\vspace{.3cm}

\years{2010--2012}\textbf{Birmingham Autonomous Robot Club}\\[.2cm]
Founded the robot club in my second year as a way to get interested
students and academics working together to make intelligent robots. It
was immensely fun and we were able to show off our robots at school
events. I also learned a lot about working on long term projects with
transient teams.

%% ------------------------------------------------------------------%%
%% IT & Programming Skills                                           %%
%% ------------------------------------------------------------------%%
\section*{IT \& Programming Skills}
Programming languages (Java, Common Lisp, Prolog, C/C++).\\
Scripting languages (Python, shell).\\
Markup languages (HTML, CSS, XML, YAML, JSON).\\
Qery languages (SQL).\\
Data analysis (Matlab).\\
Revision control (Git, Subversion).\\
Digital typesetting (\TeX, \LaTeX, \XeTeX).

%% ------------------------------------------------------------------%%
%% Languages                                                         %%
%% ------------------------------------------------------------------%%
\section*{Languages}
\textit{English} (native speaker)\\
\textit{Spanish} (conversational fluency)

%% ------------------------------------------------------------------%%
%% Publications                                                      %%
%% ------------------------------------------------------------------%%
%% \section*{Publications}
%% \setcounter{subsection}{0}
%% \subsection{Journal Articles}
%% \subsection{Edited Volumes}
%% \subsection{Book Chapters}
%% \subsection{Peer-reviewed conference papers}
%% \subsection{Peer-reviewed conference posters}
%% \subsection{Unpublished Works}
%% \subsection{Reports \& white papers}

%% ------------------------------------------------------------------%%
%% Recent Invited Talks                                              %%
%% ------------------------------------------------------------------%%
%% \section*{Recent Invited Talks}



%% ------------------------------------------------------------------%%
%% Creative Commons Attribution                                      %%
%% ------------------------------------------------------------------%%
%% \vspace{1cm}
\vfill{}
% \hrulefill
\begin{center}
  {\scriptsize  Last updated: \today\- •\-
    % ---- PLEASE LEAVE THIS BACKLINK FOR ATTRIBUTION AS PER CC-LICENSE
    Typeset in \href{http://nitens.org/taraborelli/cvtex}{
      \fontspec{Times New Roman}\XeTeX }\\
    % ---- FILL IN THE FULL URL TO YOUR CV HERE
    \href{http://jeremiahvia.com/cv/cv.pdf}{http://jeremiahvia.com/cv}}
\end{center}
\end{document}
